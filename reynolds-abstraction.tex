\documentclass{article}
\usepackage[margin=1in]{geometry}
\usepackage{mathpartir}
\usepackage{stmaryrd}

\title{Types, Abstraction, and Parametric Polymorphism}


\begin{document}
\maketitle
This document re-presents much of Reynolds' classic \emph{Types, Abstraction, and Parametric
Polymorphism} paper in modern notation.  This is the paper that defined parametricity.

\section*{An Extended Lambda Calculus}
This section defines syntax, typing, and denotational semantics for a language that is subjectively either
slightly more general than STLC (James' opinion) or slightly more restricted than System F (Colin's
opinion).  The language is approximately STLC plus some unspecified types, or System F minus
explicit type quantification.

Syntax:
\newcommand{\cond}[3]{\ensuremath{\mathsf{if}(#1)\mathsf{then}\{#2\}\mathsf{else}\{#3\}}}
\[
    \begin{array}{lrcl}
        \textrm{Type Constants} & \kappa & \in & C \;\textrm{(includes booleans)}\\
        \textrm{Type Variables} & \tau & \in & T\\
        \textrm{Types} & \omega\in\Omega & ::= & \kappa \mid \tau \mid \omega \rightarrow \omega'
        \mid \omega \times \omega'\\
        \textrm{Variables} & x,y,z & \in & V\\
        \textrm{Constants} & k & \in & \kappa\\
        \textrm{Expressions} & e & ::= & k \mid x \mid e_1(e_2) \mid (\lambda x:\omega\ldotp e) \mid
        \langle e_1 , e_2 \rangle \mid e.1 \mid e.2 \mid \cond{e}{e_1}{e_2}
    \end{array}
\]

Type rules:
\begin{mathpar}
    \inferrule{ k \in \kappa_\omega }{ \pi\vdash k : \omega }
    \and
    \inferrule{ }{\pi\vdash x : \pi(x) }
    \and
    \inferrule{
        \pi\vdash e_1 : \omega\rightarrow\omega'\\
        \pi\vdash e_2 : \omega
    }{
        \pi\vdash e_1(e_2) : \omega'
    }
    \and
    \inferrule{
        \pi,v:\omega\vdash e : \omega'
    }{
        \pi\vdash \lambda v:\omega\ldotp e : \omega'
    }
    \and
    \inferrule{
        \pi\vdash e : \omega\\
        \pi\vdash e' : \omega'\\
    }{
        \pi\vdash\langle e,e'\rangle : \omega\times\omega'
    }
    \and
    \inferrule{
        \pi\vdash\langle e\rangle : \omega\times\omega'
    }{
        \pi\vdash\langle e.1\rangle : \omega
    }
    \and
    \inferrule{
        \pi\vdash\langle e\rangle : \omega\times\omega'
    }{
        \pi\vdash\langle e.2\rangle : \omega'
    }
    \and
    \inferrule{
        \pi\vdash b : \mathsf{bool}\\
        \pi\vdash e : \omega\\
        \pi\vdash e' : \omega
    }{
        \pi\vdash\cond{b}{e}{e'} : \omega
    }
\end{mathpar}

Next Reynolds defines denotational semantics --- an interpretation of the above language into set
theory in terms of a mapping $S$ assigning sets to type variables and a mapping $CS$ assigning sets
to the base types such that $CS(\mathsf{bool})=\{\mathsf{true},\mathsf{false}\}$.  Specifically, this is a pair of
interpretations: interpreting types as the set of elements they contain, and terms as functions from
the set of contexts to the set representing the appropriate type.

Reynolds calls the first function, denoting types as the set of their elements, $S^\#$.

\newcommand{\denote}[2]{\ensuremath{\llbracket #1 \rrbracket_{#2}}}
\[
    \begin{array}{rcl}
        \denote{\kappa}{S} & = & CS(\kappa)\\
        \denote{\tau}{S} & = & S(\tau)\\
        \denote{\omega\rightarrow\omega'}{S} & = & \denote{\omega}{S}\rightarrow\denote{\omega'}{S}\\
        \denote{\omega\times\omega'}{S} & = & \denote{\omega}{S}\times\denote{\omega'}{S}\\
    \end{array}
\]
Where $\rightarrow$ and $\times$ can be used in the meta language to note a set of functions, or the
cartesian product, respectively.

Reynolds lifts this first function again to type environments as $S^{\#\star}$:
\[
    \denote{\pi}{S} = \lambda {v\in\mathsf{dom}(\pi)}\ldotp\denote{\pi(v)}{S}
\]

Expression semantics: Reynolds defines $\mu_{\pi\omega}$ as a mapping from well-typed terms and
semantic mappings to functions from environments to set elements.  He assumes for each constant type
$\omega$ a mapping $\alpha_\omega : \kappa_\omega \rightarrow \denote{\omega}{S}$.
\newcommand{\edenote}[5]{\ensuremath{{}^{#1}_{#2}\llbracket #3\rrbracket_{#4}(#5)}}
\[
    \begin{array}{rcl}
        \edenote{\pi}{\omega}{-}{-}{-} & : & \{ e \mid \pi\vdash e : \omega \} \rightarrow
        \Pi_{S:\mathcal{S}} (\denote{\pi}{S}\rightarrow\denote{\omega}{S})\\
        \edenote{\pi}{\omega}{k}{S}{\eta} & = & \alpha_\omega\;k\\
        \edenote{\pi}{\omega}{v}{S}{\eta} & = & \eta\;v\\
        \edenote{\pi}{\omega'}{e_1(e_2)}{S}{\eta} & = &
            \edenote{\pi}{\omega\rightarrow\omega'}{e_1}{S}{\eta}(\edenote{\pi}{\omega}{e_2}{S}{\eta})\\
        \edenote{\pi}{\omega\rightarrow\omega'}{\lambda x:\omega\ldotp e}{S}{\eta} & = & f \;\textrm{such that}\;
            f\;x=\edenote{\pi,v:\omega}{\omega'}{e}{S}{\eta,v:x}
        \\
        \edenote{\pi}{\omega\times\omega'}{\langle e,e' \rangle}{S}{\eta} & = & 
            \langle
            \edenote{\pi}{\omega}{e}{S}{\eta}
            ,
            \edenote{\pi}{\omega'}{e'}{S}{\eta}
            \rangle
        \\
        \edenote{\pi}{\omega}{e.1}{S}{\eta} & = & \edenote{\pi}{\omega\times\omega'}{e}{S}{\eta}_1\\
        \edenote{\pi}{\omega'}{e.2}{S}{\eta} & = & \edenote{\pi}{\omega\times\omega'}{e}{S}{\eta}_2\\
    \end{array}
\]
This is a standard set-theoretic denotational semantics.

\section*{The Abstraction Theorem}
This section is the heart of the paper, establishing the main result.  The rest of the paper is a
mixture of connections to related work and (important!) generalizations (modulo the assumption in
Section 8 of a set-theoretic semantics for System F which Reynolds later proved could not exist).


\end{document}
